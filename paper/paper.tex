\documentclass[a4paper, twocolumn]{scrartcl}

%% Limit: 40000 chars
%% check with `detex paper.tex | wc -c`

\title{DNSKEY Management}
\author{Julien Perrochet \and Tobias Schlatter}
\date{December 12, 2012}

\begin{document}

\maketitle

\section{Introduction}
%% TODO add some blah about management and penetration.

In the following, every necessary operation for key management in
DNSSEC is described as a reminder to the reader.

\paragraph{Enabling DNSSEC} When enabling DNSSEC for a zone, the
parent zone has to establish and sign a DS record with the newly
created public-key.

\paragraph{Key Rollover} RFC4641 \cite{RFC4641} suggests that the DNSKEY
records for a zone should be changed somewhere between once every week
up to once every month. While the detailed rollover procedure is
irrelevant here, it is important to know, that the parent zone has to
update the its DS record.

\paragraph{Disabling DNSSEC} An operator might choose to discontinue
DNSSEC for a zone. The DS records (or other means of signing DNSKEYs
as we will see later) need to be removed.

\section{DNSSEC Deployment}
No root-zone signing, etc.
\subsection{Penetration}
\subsection{Trust Anchors}

\section{Operations}
\subsection{3Rs}
\subsection{Single admin}
\subsection{Multi admin}
\subsection{Disabling}

\section{Case-Study: Switzerland}

\nocite{*}
\bibliographystyle{abbrv}
\bibliography{dnssec}

\end{document}

%%% Local Variables: 
%%% mode: latex
%%% TeX-master: t
%%% End: 
